\documentclass[11pt,a4paper,leqno]{article}
\usepackage{a4wide}
\usepackage[T1]{fontenc}
\usepackage[utf8]{inputenc}
\usepackage{float, afterpage, rotating, graphicx}
\usepackage{longtable, booktabs, tabularx}
\usepackage{verbatim}
\usepackage{eurosym, calc, chngcntr}
\usepackage{amsmath, amssymb, amsfonts, amsthm, bm, delarray} 
\usepackage{caption}
\usepackage{tkz-graph}
\usetikzlibrary{arrows,positioning,snakes,shapes,shapes.multipart,patterns,mindmap,shadows}

% \usepackage[backend=biber, natbib=true, bibencoding=inputenc, bibstyle=authoryear-ibid, citestyle=authoryear-comp, maxnames=10]{biblatex}
% \bibliography{bib/hmg}

\usepackage[unicode=true]{hyperref}
\hypersetup{colorlinks=true, linkcolor=black, anchorcolor=black, citecolor=black, filecolor=black, menucolor=black, runcolor=black, urlcolor=black}
\setlength{\parskip}{.5ex}
\setlength{\parindent}{0ex}

\theoremstyle{definition}
\newtheorem{exercise}{Exercise}
\renewcommand{\theenumi}{\roman{enumi}}

% Set this counter to "first exercise of the week minus one".
\setcounter{exercise}{0}

\begin{document}

\begin{center}
    \begin{large}
        \textbf{
        Effective programming practices for economists\\
        Universität Bonn, Winter 2013/14 \\[2ex]
        Exercise 3 solution\\[2ex]
        Group XXXX\\[2ex]
        Your names here
        }
    \end{large}
\end{center}


\includegraphics{../../out/figures/figure2a_risk_mort.png}

\includegraphics{../../out/figures/figure2b_gdp_mort.png}
	
\input{../../out/tables/table1_reg_on_indicators.tex}

\input{../../out/tables/table2_first_stage_est.tex}

%\input{../../out/tables/table3_iv_est.tex}




% Example for inheritance diagram from the lecture.
%
% \begin{tiny}
%     \begin{tikzpicture}
%         \node (1) [
%             rectangle split,
%             rectangle split parts=6,
%             draw,
%             text width=6.00cm,
%             shift={(-1.25,2.0)}
%         ]
%         {
%             \nodepart{one}
%             \begin{small}
%             \textbf{AgentRiskyProspectsWithGambles}
%             \end{small}
%             \nodepart{two}
%             certainty\_equivalent\_gambles()
%             \nodepart{three}
%             gambles \textcolor{red}{[tuple]}
%             \nodepart{four}
%             certainty\_equivalent(prospects, probabilities)
%             \nodepart{five}
%             expected\_utility(prospects, probabilities)
%             \nodepart{six}
%             \_\_init\_\_(gambles)
%         };
%         \node (2) [
%             rectangle split,
%             rectangle split parts=5,
%             draw,
%             text width=4.5cm,
%             shift={(-3,-2.0)}
%         ]
%         {
%             \nodepart{one}
%             \begin{small}
%             \textbf{AgentKinkedWithGambles}
%             \end{small}
%             \nodepart{two}
%             utility(z)
%             \nodepart{three}
%             utility\_inverse(x)
%             \nodepart{four}
%             lambda\_ \textcolor{red}{[float]}
%             \nodepart{five}
%             \_\_init\_\_(lambda\_, gambles)
%         };
%         \node (3) [
%             rectangle split,
%             rectangle split parts=5,
%             draw,
%             text width=3.5cm,
%             shift={(4.25,1.0)}
%         ]
%         {
%             \nodepart{one}
%             \begin{small}
%             \textbf{Gamble}
%             \end{small}
%             \nodepart{two}
%             prospects()
%             \nodepart{three}
%             probabilities()
%             \nodepart{four}
%             name \textcolor{red}{[str]}
%             \nodepart{five}
%             \_\_init\_\_(gamble\_dict, name)
%         };
%         \draw[->] (2) to [out=150, in=162] (1);
%         \draw[->] (1) to [in=140, out=2] (3);
%     \end{tikzpicture}
% \end{tiny}

% \printbibliography 

\end{document}
